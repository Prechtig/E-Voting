\chapter{Requirements}

The requirements are categorized according to the FURPS+ model.

\section{Functional requirements}

\begin{tabular}{| l | l |}
  \hline
 \textbf{Identifier} & \textbf{ Requirement summary} \\
  \hline
  \multicolumn{2}{|l|} {\textit{Features}} \\
  \hline
  FR 1.1 & The system must allow users to vote. \\
  \hline
  FR 1.2 & The system must allow eligible users to configure an election prior to the start of it. \\
  \hline
  FR 1.3 & The system must allow eligible users to start and stop an election. \\
  \hline
  FR 1.4 & The system must allow eligible users to obtain the result of an election. \\
  \hline
  FR 1.5 & The system must allow users to confirm that their vote has been registered by the bulletin board server. \\
  \hline
  \multicolumn{2}{|l|} {\textit{Interoperability}} \\
  \hline
  FR 2.1 & The system's source must be easily understandable to wide amount of software porfessionals. \\
  \hline
  FR 2.2 & All system modules require Jolie to run?. \\
  \hline
  \multicolumn{2}{|l|} {\textit{Extendability}} \\
  \hline
  FR 3.1 & Each system module must be replaceable without changing the other modules. \\
  \hline
  FR 3.2 & System modules must communicate across well defined interfaces. \\
  \hline
  \multicolumn{2}{|l|} {\textit{Composability}} \\
  \hline
   FR 4.1 & All communication between system modules must be verifiable. \\
  \hline
  \multicolumn{2}{|l|} {\textit{Manageability}} \\
  \hline
  FR 5.1 & The system must be administrated by a set of authorities. \\
  \hline
   FR 6.1 & The system must be administrated by a set of authorities. \\
   \hline
  \multicolumn{2}{|l|} {\textit{Maintainability}} \\
  \hline
   FR 7.1 & The system source code must be well commented and documented. \\
   \hline
  \multicolumn{2}{|l|} {\textit{Security}} \\
  \hline
   FR 8.1 & The system must take measures to ensure that ones ballot is always anonymous. 		\\
   \hline

\end{tabular}

\section{Detailed Requirements}

\textbf{FR 1.1 - The system must allow users to vote} \bigskip 
The system must be able to capture votes for all eligible people, such that it is taken into account when the ballots are counted. It must be possible to overwrite ones ballot as long as the election is still in progress, by the exactly same process as voting without overwriting.\\
It requires the following input to vote:
\begin{itemize}
\item The identifier of the election option.
\item The personal identifier of the voter.
\item The password of the user.
\end{itemize}
\bigskip 
\textbf{FR 1.2 - The system must allow eligible users to configure an election prior to the beginning of it} \\
It must be possible for a few selected and trusted people to configure certain options in the system. These options include, but are not limited to, the election options, the keys used for the cryptographic parts of the system and when the elections begins and ends. \\
\textbf{FR 1.3 - The system must allow eligible users to start and stop an election}

\textbf{FR 1.4 - The system must allow eligible users to obtain the result of an election} \\

\textbf{FR 1.5 - The system must allow users to confirm that their vote has been registered by the bulletin board server.}

\textbf{FR 3.1 - The system's source must be easily understandable to wide amount of software professionals} \\
The system must be written in a way which the average software professional can easily read and understand, in order to make the system more transparent. A more transparent system is, easier to verify sine more people will be able to understand the how the system work and thus be able to verify it. \\


\textbf{FR 4.1 - All communication between system modules must be verifiable} \\
Since the system shall be used to decide how many people has voted for the different candidates and/or parties, an error in the system can have a very large impact. To increase the likelihood of the software developers catching any mistake the system must be as transparent as possible. Furthermore it should be possible to verify the communication between the modules. It is important that if one module sends a message to another module, the message is the same when received.